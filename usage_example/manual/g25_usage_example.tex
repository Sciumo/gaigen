\documentclass[10pt, a4paper]{article}
\usepackage{palatino}
\usepackage{graphics}
\usepackage{latexsym}
\usepackage{amssymb}
\usepackage{amsmath}
\usepackage{amsthm}
\usepackage{longtable}
\usepackage{color}
  
\newtheorem{theorem}{Theorem}
\newtheorem{law}{Law}
\newtheorem{assumption}{Assumption}

\raggedbottom

\newcommand{\dist}{\mbox{distance}}
\newcommand{\gav}[1]{\ensuremath{\mathbf{#1}}}
\newcommand{\bev}[1]{\ensuremath{\mathbf{e}_{#1}}}
\newcommand{\op}{\wedge}
\newcommand{\ip}{\cdot}
\newcommand{\dual}[1]{{#1}^*}
\newcommand{\gp}{\,}
\newcommand{\inv}[1]{{#1}^{-1}}
\newcommand{\rev}[1]{\widetilde{#1}}
\newcommand{\no}{\ensuremath{{\bf o}}}
\renewcommand{\ni}{\ensuremath{{\bf  \infty}}}
\renewcommand{\no}{\ensuremath{e_{\bf  o}}}
\renewcommand{\ni}{\ensuremath{e_{\bf  \infty}}}
\newcommand{\grade}[1]{\mbox{\rm grade}}
\newcommand{\gradeop}[2]{{\langle {#1}\rangle_{#2}}}
\newcommand{\NOTE}[1]{\mbox{\bf [[~{#1}~]]}}
\newcommand{\R}{\mathbb{R}}
\newcommand{\half}{{\mbox{$\frac{1}{2}$}}}

\definecolor{black}{gray}{0}
\definecolor{dark}{gray}{0.5}
\definecolor{red}{rgb}{1,0,0}
\definecolor{darkred}{rgb}{0.65,0,0}
\definecolor{blue}{rgb}{0,0,1}
\definecolor{darkgreen}{rgb}{0,0.65,0}
\newcommand{\tcb}[1]{\textcolor{blue}{#1}}
\newcommand{\tcr}[1]{\textcolor{red}{#1}}
\newcommand{\tcdr}[1]{\textcolor{darkred}{#1}}
\newcommand{\tcd}[1]{\textcolor{dark}{#1}}
\newcommand{\tcdg}[1]{\textcolor{darkgreen}{#1}}

\begin{document}

\title{Gaigen 2.5 Usage Example}
\author{
Daniel Fontijne\\EUVISION Technologies
}
\date{\today}
\maketitle

\section{Introduction}

This document describes the small example of usage that is provided for each supported language.

The usage example does the following:
\begin{itemize}
\item Create five points.
\item Use the outer product to form a line and a plane out of the point.
\item Compute the intersection of the line and the plane.
\item Print the computed values to the console.
\end{itemize}

Makesfiles and/or a IDE projects for each major platform. For Java,
an Eclipse project is provided. For the other languages, Visual Studio projects 
are provided for C, C++ and C-Sharp on Windows. For Linux and OS X, these three
languages are covered by makefiles.

For most languages, the usage example is split into four steps. Each of these steps implements
the above functionality, but differ in the stage of optimization. Step 1 is not optimized
at all, but flexible in its usage of variables, while step 4 is fully optimized. Step 2 and
3 are intermediate steps. The steps are:

\begin{enumerate}
\item All geometric variables are stored in general multivectors (type `mv'). While this is not
very time- and storage- efficient, it is more flexible.
\item Geometric variables are stored in specialized multivectors (like `normalizedPoint' and `line'). 
During computations (like the outer product), these variables are converted to general multivectors. 
The results of the computations are converted back to a specialized type.
\item This step shows how to automatically extract what functions to optimize using the 
`reportUsage' functionality. This is an iterative process. You have to go back and forth between
compiling and editing the specification XML file.
\item Geometric variables are stored in specialized multivectors (like `normalizedPoint' and `line'). 
Functions over these variables are fully optimized.
\end{enumerate}

Personally, I always jump straight to step 4 and use the `reportUsage' feature of step 3 only to 
verify that I have not forgotten an optimization. I write the specification (.xml) files along
with the rest of the program, adding new functions and specialized types as needed.

What follows are language-specific notes.

\section{C}

In general, use the C language version only when there is no other option,
or for wrapping it into another language (like Matlab).

In C, there is no function overloading, so you are fully aware of exactly what function 
is being called with what arguments. Thus step 2 and 3 make no sense and have been left out.

In step 1, note that the function cgaPoint() always returns a normalizedPoint. 
It needs to be explicitlyed convert to the mv type.

At the end of step 2, an example is given of how to convert the flatPoint type to a 
normalizedPoint (a common annoyance in conformal GA). Note that this example uses a 
Euclidean vector type internally.

\section{C++}
 TODO
 
\section{C-Sharp }

TODO

\section{Java}

TODO

\end{document}
