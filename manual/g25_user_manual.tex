\documentclass[10pt, a4paper]{article}
\usepackage{palatino}
\usepackage{graphics}
\usepackage{latexsym}
\usepackage{amssymb}
\usepackage{amsmath}
\usepackage{amsthm}
\usepackage{longtable}
\usepackage{color}
  
\newtheorem{theorem}{Theorem}
\newtheorem{law}{Law}
\newtheorem{assumption}{Assumption}

\raggedbottom

\newcommand{\dist}{\mbox{distance}}
\newcommand{\gav}[1]{\ensuremath{\mathbf{#1}}}
\newcommand{\bev}[1]{\ensuremath{\mathbf{e}_{#1}}}
\newcommand{\op}{\wedge}
\newcommand{\ip}{\cdot}
\newcommand{\dual}[1]{{#1}^*}
\newcommand{\gp}{\,}
\newcommand{\inv}[1]{{#1}^{-1}}
\newcommand{\rev}[1]{\widetilde{#1}}
\newcommand{\no}{\ensuremath{{\bf o}}}
\renewcommand{\ni}{\ensuremath{{\bf  \infty}}}
\renewcommand{\no}{\ensuremath{e_{\bf  o}}}
\renewcommand{\ni}{\ensuremath{e_{\bf  \infty}}}
\newcommand{\grade}[1]{\mbox{\rm grade}}
\newcommand{\gradeop}[2]{{\langle {#1}\rangle_{#2}}}
\newcommand{\NOTE}[1]{\mbox{\bf [[~{#1}~]]}}
\newcommand{\R}{\mathbb{R}}
\newcommand{\half}{{\mbox{$\frac{1}{2}$}}}

\definecolor{black}{gray}{0}
\definecolor{dark}{gray}{0.5}
\definecolor{red}{rgb}{1,0,0}
\definecolor{darkred}{rgb}{0.65,0,0}
\definecolor{blue}{rgb}{0,0,1}
\definecolor{darkgreen}{rgb}{0,0.65,0}
\newcommand{\tcb}[1]{\textcolor{blue}{#1}}
\newcommand{\tcr}[1]{\textcolor{red}{#1}}
\newcommand{\tcdr}[1]{\textcolor{darkred}{#1}}
\newcommand{\tcd}[1]{\textcolor{dark}{#1}}
\newcommand{\tcdg}[1]{\textcolor{darkgreen}{#1}}

\begin{document}

\title{Gaigen 2.5 User Manual}
\author{
Daniel Fontijne\\University of Amsterdam 
}
\date{\today}
\maketitle

\section{Introduction}

Gaigen 2.5 is a code generator for geometric algebra. It compiles an XML specification
of an algebra into an implementation. Supported output languages are C and C++.
Support for C\# and Java will be finished in May 2010. The tool itself is written in C\#.

\section{Some History}

Gaigen stands for \emph{Geometric Algebra Implementetion GENerator}.

The first version of Gaigen was written in 2001. It supported only C++ as output language
and the performance of the generated code was two to five times slower than the equivalent
use of linear algebra. At that time it was the fastest general purpose GA implemention available. 
Gaigen 1 supported only one multivector type and used coordinate  compression and profiling to increase performance.

The second version of Gaigen was written in 2006. It added support for specialized multivector
types and the Java. The generated code was competative with linear algebra (faster for some problems,
slower for others).

This version (2.5) is a re-implementation of Gaigen 2. It removes a lot of 'dead weight' and
primarily aims at making Gaigen suitable for a production environment, with an emphasis on
testing, programming language support, scalability and extendability.

See \cite{GA4CS} and \cite{fontijnePhD} for background information on (implementation of) geometric algebra.


\section{Installing}

An installer is provided for each platform.
One Linux and OSX, Mono is required because Gaigen 2.5 is written in C\#.
Mono is a free, open source implementation of Microsofts .Net platform.

\begin{itemize}
\item[Windows] Run {\tt InstallG25.msi}.
\item[Linux] Make sure Mono and ANTLR are installed, then install the RPM using 
{\tt rpm -i g25-2.5.X.rpm}. Use {\tt rpm -U g25-2.5.X.rpm} to update.
By default three small scripts are installed in {\tt /usr/bin}, the actual program is installed in {\tt /usr/share/g25}.
\item[Mac OS X] Install Mono, then run {\tt g25.pkg}. Three small scripts are installed in {\tt /usr/bin}, the actual 
program is installed in {\tt /usr/share/g25}.
\end{itemize}

ANTLR is required to compile grammar for parsing multivector strings.
This is optional, as a built-in parser is also included.

\section{Running}

To compile an algebra specification, run {\tt g25 specfile.xml} from the command line.
	
The first time you run {\tt g25} it may be a bit slow because it is compiling
a lot of code templates on the fly. Compiled templates are stored in a temporary 
directory and recycled in future runs.
On Linux and OSX, can take a quite some time due to Mono.

	
The following command line options are available:

{\tt -h -help -?}: display help.
Example: {\tt g25 -h}

{\tt -v, -version}: display version information.
Example: {\tt g25 -v}

{\tt -s, -save}: read specification file, then save it.
Example: {\tt g25 -s spec_out.xml spec_in.xml}

{\tt -f, -filelist}: save a list of generated files to a file.
Example: {\tt g25 -f filelist.txt spec.xml}

\section{Sample Algebras}

Use the following command to generate a few specification of algebras that may be used
as a starting for your own specs: {\tt g25_test_generator -sa }.
This creates a directory {\tt TestG25}. In that directory will be four specifications for
algebras, for each supported language. Scripts to build and test the algebras are also
included.

The sample algebras are:
\begin{itemize}
\item[e2ga] 2-D Euclidean geometric algebra.
\item[e3ga] 3-D Euclidean geometric algebra.
\item[p3ga] Geometric algebra for the homogeneous model of 3-D space (so it is a 4-D algebra).
Multivector types for points, planes, lines and so on are defined.
\item[c3ga] Geometric algebra for the conformal model of 3-D space (so it is a 4-D algebra).
Multivector types for points, spheres, circles, lines and so on are defined.
\end{itemize}



\section{Why did you write Gaigen 2.5 C\#?}

I wrote Gaigen 2.5 in C\# because I wanted to try out the platform.
I assumed that Mono would be good enough to run Gaigen 2.5 on OS X and
Linux, but it is slightly disappointing in speed. Hopefully this will
situation will improve in the future.


\begin{thebibliography}{}


	
\bibitem{GA4CS} L.~Dorst and D.~Fontijne and S.~Mann.
	\emph{Geometric Algebra for Computer Science: An Object Oriented Approach to Geometry.}
	Morgan Kaufmann, revised edition 2009.

\bibitem{fontijnePhD} D.~Fontijne.
	\emph{Efficient Implementation of Geometric Algebra.}
	PhD. Thesis, University of Amsterdam, 2007.


\end{thebibliography}


\end{document}
