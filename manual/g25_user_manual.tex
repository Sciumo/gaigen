\documentclass[10pt, a4paper]{article}
\usepackage{palatino}
\usepackage{graphics}
\usepackage{latexsym}
\usepackage{amssymb}
\usepackage{amsmath}
\usepackage{amsthm}
\usepackage{longtable}
\usepackage{color}
  
\newtheorem{theorem}{Theorem}
\newtheorem{law}{Law}
\newtheorem{assumption}{Assumption}

\raggedbottom

\newcommand{\dist}{\mbox{distance}}
\newcommand{\gav}[1]{\ensuremath{\mathbf{#1}}}
\newcommand{\bev}[1]{\ensuremath{\mathbf{e}_{#1}}}
\newcommand{\op}{\wedge}
\newcommand{\ip}{\cdot}
\newcommand{\dual}[1]{{#1}^*}
\newcommand{\gp}{\,}
\newcommand{\inv}[1]{{#1}^{-1}}
\newcommand{\rev}[1]{\widetilde{#1}}
\newcommand{\no}{\ensuremath{{\bf o}}}
\renewcommand{\ni}{\ensuremath{{\bf  \infty}}}
\renewcommand{\no}{\ensuremath{e_{\bf  o}}}
\renewcommand{\ni}{\ensuremath{e_{\bf  \infty}}}
\newcommand{\grade}[1]{\mbox{\rm grade}}
\newcommand{\gradeop}[2]{{\langle {#1}\rangle_{#2}}}
\newcommand{\NOTE}[1]{\mbox{\bf [[~{#1}~]]}}
\newcommand{\R}{\mathbb{R}}
\newcommand{\half}{{\mbox{$\frac{1}{2}$}}}

\definecolor{black}{gray}{0}
\definecolor{dark}{gray}{0.5}
\definecolor{red}{rgb}{1,0,0}
\definecolor{darkred}{rgb}{0.65,0,0}
\definecolor{blue}{rgb}{0,0,1}
\definecolor{darkgreen}{rgb}{0,0.65,0}
\newcommand{\tcb}[1]{\textcolor{blue}{#1}}
\newcommand{\tcr}[1]{\textcolor{red}{#1}}
\newcommand{\tcdr}[1]{\textcolor{darkred}{#1}}
\newcommand{\tcd}[1]{\textcolor{dark}{#1}}
\newcommand{\tcdg}[1]{\textcolor{darkgreen}{#1}}

\begin{document}

\title{Gaigen 2.5 User Manual}
\author{
Daniel Fontijne\\University of Amsterdam 
}
\date{\today}
\maketitle

\section{Introduction}

Gaigen 2.5 is a code generator for geometric algebra. It compiles an XML specification
of an algebra into an implementation. Supported output languages are C and C++.
Support for C\# and Java will be finished in May 2010. The tool itself is written in C\#.

\section{History and Background}

Gaigen stands for \emph{Geometric Algebra Implementation GENerator}.

The first version of Gaigen was written in 2001. It supported only C++ as output language
and the performance of the generated code was two to five times slower than the equivalent
use of linear algebra. At that time it was the fastest general purpose GA implementation available. 
Gaigen 1 supported only one multivector type and used coordinate  compression and profiling to increase performance.
The tool itself was written in C++.

The second version of Gaigen was written in 2006. It added support for specialized multivector
types and supported C++ and Java. The tool itself was written in Java.
The generated code was competative with linear algebra (faster for some problems, slower for others).

This version (2.5) is a re-implementation of Gaigen 2. It removes a lot of 'dead weight' and
primarily aims at making Gaigen suitable for a production environment, with an emphasis on
testing, programming language support, scalability and extendability.

See \cite{GA4CS} and \cite{fontijnePhD} for background information on (implementation of) geometric algebra.


\section{Installing}

An installer is provided for each platform.
One Linux and OSX, Mono is required because Gaigen 2.5 is written in C\#.
Mono is a free, open source implementation of Microsofts .Net platform.

\begin{itemize}
\item {\bf Windows: } Run {\tt InstallG25.msi}. By default, the files are installed in {\tt c:\\program files\\g25}.
The system path is updated to include this directory.
\item {\bf Linux: } Make sure Mono and ANTLR are installed, then install the RPM using 
{\tt rpm -i g25-2.5.X.rpm}. Use {\tt rpm -U g25-2.5.X.rpm} to update.
By default three small scripts are installed in {\tt /usr/bin}, the actual program is installed in {\tt /usr/share/g25}.
\item {\bf Mac OS X:} Install Mono, then run {\tt g25.pkg}. Three small scripts are installed in {\tt /usr/bin}, the actual 
program is installed in {\tt /usr/share/g25}.
\end{itemize}

ANTLR is only required to compile grammars for parsing multivector strings.
To avoid its use, simply ask for the built-in parser in the specification.

\section{Running}

To compile an algebra specification, run {\tt g25 specfile.xml} from the command line.
	
The first time you run {\tt g25} it may be a bit slow because it is compiling
a lot of code templates on the fly. Compiled templates are stored in a temporary 
directory and recycled in future runs.
On Linux and OSX, the first run can take a even more time because Mono is a bit
slower than Microsofts CLR implementation.

	
The following command line options are available:

\vspace*{2mm}

\noindent {\tt -h -help -?}: display help.\\
Example: {\tt g25 -h}

\vspace*{2mm}

\noindent {\tt -v, -version}: display version information.\\
Example: {\tt g25 -v}

\vspace*{2mm}

\noindent {\tt -s, -save}: read specification file, then save it.\\
Example: {\tt g25 -s spec\_out.xml spec\_in.xml}

\vspace*{2mm}

\noindent {\tt -f, -filelist}: save a list of generated files to a file.\\
Example: {\tt g25 -f filelist.txt spec.xml}

\section{Sample Algebras}
\label{s:sample_algebras}

Use the following command to generate a few algebra specifications that may be used
as a starting for your own specs: {\tt g25\_test\_generator -sa }.
This creates a directory {\tt TestG25}. In that directory will be four algebra specifications.
Make files, and cripts to build and test the algebras are also included, for each supported
language.

The sample algebras are:
\begin{itemize}
\item {\bf e2ga} 2-D Euclidean geometric algebra.
\item {\bf e3ga} 3-D Euclidean geometric algebra.
\item {\bf p3ga} Geometric algebra for the homogeneous model of 3-D space (so it is a 4-D algebra).
Multivector types for points, planes, lines and so on are defined.
\item {\bf c3ga} Geometric algebra for the conformal model of 3-D space (so it is a 5-D algebra).
Multivector types for points, spheres, circles, lines and so on are defined.
\end{itemize}

\section{Documentation of Generated Code}

The generated code is self-documenting, although this
feature is not fully finished yet. Run {\tt doxygen } to extract
the documentation from the code.

The generated test code is also a good way to get an example
of use of each function.

\section{Testing the Generated Code}

Gaigen 2.5 can generate a test suite with each algebra.
This is controlled via an option in the specification file.

The sample algebras (Section~\ref{s:sample_algebras}) generate the test 
code by default. Building them will result in an executables named 
{\tt test } which try to test whether the code is working correctly. 


\subsection{Testing of Gaigen 2.5}

A special tool called {\tt g25\_test\_generator} is included with Gaigen 2.5.
Its purpose is to generate variations on algebra specifications for thorough testing
Gaigen 2.5. It is used during development.

If you are interested in running it, run {\tt g25\_test\_generator -r 5000 -s} on
the command line.
This should generate a directory named {\tt TestG25}. 
Inside that directory, run the {\tt build} script to see if all the
generated algebras actually build. Run the {\tt test} script to see if
all algebras work correctly. Run the {\tt xml\_test} script to see if
loading and saving specification XML files works correctly.
Run the {\tt clean} script to get rid of all intermediate build and test
files.

The {\tt -r} option reduces the number of algebras generated. Without it,
many thousands of algebras would be generated, which would take a long time to build.
Building takes a long time anyway since the test algebras include an unrealistic
large number of functions.

The command line options to {\tt g25\_test\_generator} are:
\vspace*{2mm}

\noindent {\tt -r -reduce }: reduce the number of test algebras by approximately that factor.
Example: {\tt g25\_test\_generator -r 5000}

\vspace*{2mm}

\noindent {\tt -s -shuffle}: shuffle the order in which the test algebras are built, tested.
Example: {\tt g25\_test\_generator -s}

\vspace*{2mm}

\noindent {\tt -sa -sample\_algebras}: generate the sample algebras instead of the randomly
selected test algebras.
Example: {\tt g25\_test\_generator -sa}

\section{Building from source}

To build Gaigen 2.5 from source, first download the source code as a tarball {\tt g25-2.5.X.tar.gz}
or from SourceForge SVN.

Using Visual Studio or MonoDevelop, open the main project {\tt g25.sln} in {\tt g25/vs2008}.
(When using MonoDevelop you will get a warning that the Windows installer project cannot be loaded.)
Do {\tt Build->Build Solution} to build all. 

If you want to build from the command line on Windows, open a Visual Studio Command Prompt and
go to the directory {\tt g25/vs2008}. Do {\tt msbuild g25.sln /p:Configuration=Release}.

If you want to build from the command line on Linux or OSX, go to the directory {\tt g25/vs2008}
and do 
\begin{verbatim}
export MONO_IOMAP=all  # makes mono tools case and slash insensitive
cd g25/vs2008
xbuild g25.sln /p:Configuration=Release
cd ../../g25_diff/vs2008
xbuild g25_diff.csproj /p:Configuration=Release
cd ../../g25_test_generator/vs2008
xbuild g25_test_generator.csproj /p:Configuration=Release
\end{verbatim}

Installers / packagers for each platform are in {\tt g25\\setup\_win}, 
{\tt g25/setup\_osx} and {\tt g25/setup\_linux}.


\section{XML file format}

A good starting point for writing you own specifications are the sample algebras
(Section~\ref{s:sample_algebras}). The format of the specification file is described
here.




TODO
Take (remove) from code.

\section{Why did you write Gaigen 2.5 C\#?}

I wrote Gaigen 2.5 in C\# because I wanted to try out the platform.
I assumed that Mono would be good enough to run Gaigen 2.5 on OS X and
Linux, but it is slightly disappointing in speed. Hopefully the Mono
team will improve Mono's performance in the future.




\begin{thebibliography}{}


	
\bibitem{GA4CS} L.~Dorst and D.~Fontijne and S.~Mann.
	\emph{Geometric Algebra for Computer Science: An Object Oriented Approach to Geometry.}
	Morgan Kaufmann, revised edition 2009.

\bibitem{fontijnePhD} D.~Fontijne.
	\emph{Efficient Implementation of Geometric Algebra.}
	PhD. Thesis, University of Amsterdam, 2007.


\end{thebibliography}


\end{document}
